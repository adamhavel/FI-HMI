\documentclass[12pt]{article}
\usepackage{amsmath}
\usepackage[numbers]{natbib}
\usepackage[czech]{babel}
\usepackage[T1]{fontenc}
\usepackage[utf8]{inputenc}
\title{FI-HMI: Temperovaná ladění}
\date{\today}
\author{Adam Havel}
\begin{document}

\maketitle

\section{Kapitola}
1. Hudební sluch

2. Harmonická řada (vibrace, barva nástroje)

3. Pentatonika (šestý alikvótní tón, oktávy)

4. Přirozené ladění (Pythagorejské, Didymické, enharmonické intervaly, kvintový kruh)

5. Baroko a cembalo (modulace, Bach, dur-mollové harmonie, konsonantní tercie a kvintakordy)

6. Temperovaná ladění (rozdílnost tónin)

7. Rovnoměrně temperované ladění (důvod, důsledky, matika)

8. Současnost

\cite{bernstein} \cite{smolka}

\begin{align}
E &= mc^2                              \\
m &= \frac{m_0}{\sqrt{1-\frac{v^2}{c^2}}}
\end{align}

\bibliography{sources}{}
\bibliographystyle{plainnat}

\end{document}