\documentclass[12pt]{article}
\usepackage{amsmath}
\usepackage{url}
\usepackage[numbers]{natbib}
\usepackage[czech]{babel}
\usepackage[T1]{fontenc}
\usepackage[utf8]{inputenc}
\title{FI-HMI: Temperovaná ladění}
\date{\today}
\author{Adam Havel}
\begin{document}

\maketitle

\section{Kapitola}
Každý alespoň trochu schopný muzikant by měl disponovat dobrým relativním sluchem, tedy schopností poznat frekvenční rozdíl mezi dvěma znějícími tóny a jen na základě poslechu rozhodnout, jaký hudební interval je odděluje. Takových intervalů je alespoň v naší klasické hudební nauce dvanáct, což je přirozeně stejné množství, jaké nám nabízí repertoár hudebních tónů neboli chromatická řada. A právě umění rozpoznat krom intervalu i samotné absolutní výšky, to znamená přiřadit je k jednomu z těchto dvanácti tónů, jako třeba C nebo G$\flat$, se nazývá absolutním sluchem.

Většina lidí má představu, že je tato schopnost vrozená a spojuje si ji s jakýmsi přirozeným hudebním nadáním — ostatně jako příklad člověka, který byl tímto sluchem obdařen, se často udává třeba Wolfgang Amadeus Mozart. Jak ovšem zjistíme, s tímto \uv{Svatým grálem} muzikantů to není tak úplně jednoduché.

Dnes už tušíme, že absolutní sluch se formuje především v raném dětství a hlavní faktor představuje opakované vystavení hudebním tónům spolu s dalším podnětem, třeba vizuálním, který ke konkrétním tónům přiřadí dané označení. Dětský mozek je pak schopen toto spojení zachovat a slyší-li pak jedinec, u kterého se tato vlastnost projeví, správnou frekvenci, na mysl mu okamžitě vytane i název tónu.

Problém ovšem nastane v momentě, kdy si uvědomíme, že frekvence jednotlivých tónů jsou jen předmětem úzu a momentální dohody a jejich hodnoty se tedy v průběhu času významně mění. Dnes se většinou jako reference pro lazení nástrojů používá tzv. koncertní A o hodnotě 440 Hz, nicméně to samé A bylo na počátku předchozího století skoro o deset herzů menší. Podíváme-li se až do osmnáctého století, zjistíme dokonce, že se tato frekvence významně lišila nejen v čase, ale například i v rámci různých měst, a to i o desítky herzů \cite{wiki_pitch}!

Ve světle těchto zjištění se absolutní sluch jeví jako svého druhu Danajský dar. Jsme-li schopni rozeznat tóny podle dnešních měřítek, 

\pagebreak

Harmonická řada (vibrace, barva nástroje)

Pentatonika (šestý alikvótní tón, oktávy)

Přirozené ladění (Pythagorejské, Didymické, enharmonické intervaly, kvintový kruh)

Baroko a cembalo (modulace, Bach, dur-mollové harmonie, konsonantní tercie a kvintakordy)

Temperovaná ladění (rozdílnost tónin)

Rovnoměrně temperované ladění (důvod, důsledky, matika)

Současnost

\begin{align}
E &= mc^2                              \\
m &= \frac{m_0}{\sqrt{1-\frac{v^2}{c^2}}}
\end{align}

\bibliography{sources}{}
\bibliographystyle{plainnat}

\end{document}